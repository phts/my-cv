\documentclass[a4paper, 12pt]{article}
\usepackage[cm]{fullpage}
\usepackage[nomessages]{fp}
\usepackage[utf8]{inputenc}

\usepackage{hyperref}
\hypersetup{
    pdftitle={My CV},
    pdfauthor={Phil Tsaryk},
    pdfcreator={Phil Tsaryk},
    colorlinks=true,
    urlcolor=blue
}

\newcommand{\position}[1]{
    % bold
    \textbf{#1}}
\newcommand{\itemlabel}[1]{
    % italic + ":"
    \textit{#1:}}
\newcommand{\lastmodified}{
    \tiny{\itemlabel{Last Modified} \today}}
\newcommand{\defvalue}[2]{\ifx#1\empty#2\else#1\fi}
\newcommand{\yearsexp}{%
    \FPeval{\result}{clip(\the\year - 2011)}%
    \defvalue{\result}{6+}}

% Switch implementation
\usepackage{xifthen}
\newcommand{\ifequals}[3]{\ifthenelse{\equal{#1}{#2}}{#3}{}}
\newcommand{\case}[2]{#1 #2} % Dummy, so \renewcommand has something to overwrite...
\newenvironment{switch}[1]{\renewcommand{\case}{\ifequals{#1}}}{}

\newcommand{\monthname}[1]{%
    \begin{switch}{#1}%
        \case{01}{Jan}%
        \case{02}{Feb}%
        \case{03}{Mar}%
        \case{04}{Apr}%
        \case{05}{May}%
        \case{06}{Jun}%
        \case{07}{Jul}%
        \case{08}{Aug}%
        \case{09}{Sep}%
        \case{10}{Oct}%
        \case{11}{Nov}%
        \case{12}{Dec}%
    \end{switch}%
}
\newcounter{datediffyears}%
\newcounter{datediffmonths}%
\newcommand{\datediff}[4]{%
    \setcounter{datediffyears}{#3}%
    \setcounter{datediffmonths}{#4}%
    \addtocounter{datediffmonths}{-#2 + 1}%
    \addtocounter{datediffyears}{-#1}%
    \ifnum\thedatediffmonths<0%
        \setcounter{datediffmonths}{12 + \thedatediffmonths}%
        \addtocounter{datediffyears}{-1}%
        \addtocounter{datediffmonths}{\thedatediffyears * 12}%
    \else%
        \addtocounter{datediffmonths}{\thedatediffyears * 12}%
    \fi
    \the\numexpr\thedatediffmonths\relax\space months%
}
\newcommand{\difftoday}[2]{%
    \monthname{#2} #1--Present (\datediff{#1}{#2}{\the\year}{\the\month})%
}
\newcommand{\diffdates}[4]{%
    \monthname{#2} #1--\monthname{#4} #3 (\datediff{#1}{#2}{#3}{#4})%
}

% Write current page number and Last Modified date in the footer
\usepackage{fancyhdr}
\pagestyle{fancyplain}
\fancyhf{}
\renewcommand{\headrulewidth}{0pt} % remove header line
\cfoot{\thepage}
\rfoot{\lastmodified}

\title{Phil Tsaryk}
\author{}
\date{}

\begin{document}
\maketitle

I am highly motivated person with \yearsexp{} years experience in a front-end development.
I have solid knowledge in modern web technologies including ES6+, React, TypeScript, Angular, Node.js, etc.
I have broad experience in conforming JavaScript and HTML/CSS code to work in old browsers.
Familiar with mobile first approach, making responsive design.
Experienced in full stack development using Ruby, Ruby on Rails, Java, Node.js.

I am experienced in designing solutions from scratch and establishing development processes on the project, including CI process.
Experienced in supporting legacy code with further step-by-step refactoring.
I have strong knowledge of design patterns and web platform specifics.

I have a very good feedback from customers, team members and managers.
Enthusiastic about challenging tasks, fast learner, good team player with strong motivation and good communication skills.
Always ready to participate in mentoring programs as a mentee and a mentor.

\subsubsection*{Contacts:}
\begin{itemize}
    \item Tel: \href{tel:+48534898187}{+48 534 898 187}
    \item Email: \href{mailto:phil.tsarik@gmail.com}{phil.tsarik@gmail.com}
    \item Skype: \href{callto:phil.tsarik}{phil.tsarik}
    \item My legal name: Pilip Tsaryk
\end{itemize}


\section*{Skill Overview}

    \begin{itemize}
        \item \itemlabel{Front end} HTML5, CSS3, JavaScript (incl. ES6+), TypeScript, Sass, Less, Angular (v1.x and v2+), React (and related technologies like Redux, MobX, Reflux, Preact, Styled-components, etc.)
        \item \itemlabel{Back end} Ruby, Python, Bash, Node.js, GraphQL, MySQL, PostgreSQL
    \end{itemize}


\section*{Experience}

    \begin{itemize}

        \item \position{Front-End Developer}, \difftoday{2018}{04}

            Developing marketing landing pages on HTML5/CSS3/ES6.
            Participating in a constant improving of development process and getting rid of legacy stuff.

            Developed and supporting an interactive tool (distributed as a NPM package) which allows to create a working template of a landing page.
            This is Node.js application uses EJS templating engine to simplify the process of gathering parts of a page together and to automate a routine and manual work.

            Also involved to planning and developing of a new application from scratch - a graphical constructor for non-tech guys for producing and publishing landing pages.
            This is React application generates a static page using Gatsby and is bound with multiple internal services.

            \begin{itemize}
                \item \itemlabel{Company} Grand Parade, Kraków Area, Poland
                \item \itemlabel{Tools \& technologies used} HTML5/CSS3, JavaScript, SASS, React, Gatsby, EJS, Webpack, rollup, mocha, Git, jira, Jenkins, Node.js
            \end{itemize}

        \item \position{JavaScript Developer}, \diffdates{2017}{12}{2018}{03}

            Developing an application which is being used by 1M+ users every day.

            I am responsible for UI development. All code is covered with BDD tests, unit-tests, integration tests.

            Development process consists of Kanban methodology, merge requests for every story/task, code reviews of merge requests, CI to run tests and make builds.

            In the same time supporting another project from the same domain. It is React application with GraphQL and other modern stuff.

            \begin{itemize}
                \item \itemlabel{Company} EPAM Systems, Kraków Area, Poland
                \item \itemlabel{Tools \& technologies used} JavaScript, React, Reflux, SASS, Mustache templates, GraphQL, Apollo, Webpack, Grunt, Karma, Jasmine, Git, Jira, Jenkins
            \end{itemize}

        \item \position{JavaScript Developer}, \diffdates{2015}{11}{2017}{11}

            Developing a web-application for managing some specific hierarchy data which allows adding, removing, sorting, merging, exporting to Excel files, etc.

            I am responsible for UI development. JavaScript code with business logic is covered by unit tests. I also perform code reviews of my collegues.

            I take part in daily standups and weekly demos with the customer. I had a chance to show features invented by me from scratch to the customer.

            \begin{itemize}
                \item \itemlabel{Company} EPAM Systems, Kraków Area, Poland
                \item \itemlabel{Tools \& technologies used} JavaScript, ES6, AngularJS 1.6, Redux, Bootstrap, Lodash, Webpack, Gulp, Bower, Karma, Jasmine, Protractor, Git, Jira, TeamCity
            \end{itemize}

        \item \position{Front-End Developer}, \diffdates{2016}{07}{2017}{06}

            Taking part in a friend's startup as a front-end developer.
            This is a social network-like application for people in a specific area of business and it works together with some hardware devices.

            Responsible for a whole front-end part.

            \begin{itemize}
                \item \itemlabel{Tools \& technologies used} Angular 2, Material design, TypeScript, Lodash, Node.js, Webpack, Karma, Jasmine, Protractor, Git, Bitbucket, Jenkins, AWS
            \end{itemize}

        \item \position{JavaScript Developer}, \diffdates{2014}{04}{2015}{10}

            Developing a client application for monitoring remote servers including separate versions for desktop and mobile browsers.

            \begin{itemize}
                \item \itemlabel{Company} SaM Solutions Gmbh, Minsk, Belarus
                \item \itemlabel{Responsibilities} Planning, UI development, manual testing, taking part in meetings, UI design, making mockups, reporting
                \item \itemlabel{Tools \& technologies used} JavaScript, Sencha Ext JS, Sencha Touch, SASS, SCCI protocol, Sencha Architect
            \end{itemize}

        \item \position{Full Stack Developer}, \diffdates{2012}{10}{2014}{02}

            Developing a firmware with Web GUI for network-attached storages based on Debian GNU/Linux.

            Developed and released a complete product with periodical updates with fixes and improvements.

            \begin{itemize}
                \item \itemlabel{Company} SaM Solutions Gmbh, Minsk, Belarus
                \item \itemlabel{Responsibilities} Planning, UI and back-end development, writing automated tests, manual testing, taking part in meetings, architecture and UI design, making mockups, collaborating with translators, reporting
                \item \itemlabel{Tools \& technologies used} Debian, Ruby, Sinatra, Rspec, Haml, HTML, CSS, JavaScript, jQuery, Dropbox SDK, Jasmine, Git, Jenkins, Transifex, Jira
            \end{itemize}

        \item \position{Java Developer}, \diffdates{2011}{11}{2012}{09}

            Developing client-server application which provides remote management and configuration of servers.

            \begin{itemize}
                \item \itemlabel{Company} SaM Solutions Gmbh, Minsk, Belarus
                \item \itemlabel{Responsibilities} Planning, developing, testing, reporting
                \item \itemlabel{Tools \& technologies used} Java SE, Java RMI, Swing, JAXB, Windows PowerShell, SVN
            \end{itemize}

        \item \position{Full Stack Developer}, \diffdates{2011}{02}{2012}{08}

            Taking part in developing of a language learning service \href{http://langaroo.com}{Langaroo}.

            \begin{itemize}
                \item \itemlabel{Responsibilities} UI and back-end development, writing automated tests, manual testing
                \item \itemlabel{Tools \& technologies used} Ruby, Ruby on Rails, PostgreSQL, ERB, SASS, Compass, HTML, CSS, JavaScript, jQuery, Git
            \end{itemize}

    \end{itemize}


\section*{Education}

    \begin{itemize}

        \item \position{Polotsk State University}, 2006--2011

            Diploma in Software Engineering.

    \end{itemize}

\section*{Appendix}

    \begin{itemize}
        \item \href{https://github.com/phts}{https://github.com/phts}
        \item \href{https://www.linkedin.com/in/tsaryk}{https://www.linkedin.com/in/tsaryk}
        \item \href{https://stackoverflow.com/users/2462524/phts}{https://stackoverflow.com/users/2462524/phts}
        \item \href{http://tsarik.me/}{http://tsarik.me/}
    \end{itemize}

\end{document}
